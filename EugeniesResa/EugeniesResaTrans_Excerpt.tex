% Header information
% Stipulates that the document is a book
\documentclass[12pt]{article}
% Includes utf-8 encoding
\usepackage[utf8]{inputenc}
% Package for presenting dual columns
\usepackage{parcolumns}
% Package for handling multiple languages in a single document
\usepackage{polyglossia}
% Hyperref package for generating hyperlinks
\usepackage{hyperref}

% Title to appear at head of document
\title{Excerpt from \textit{The Voyage of the Frigate Eugenie Around the World during the years 1851-1853}}
% Indicate author and translator; a more elegant way of doing this is needed
% Currently don't know how to indicate a role of editor/translator
\author{
C. Skogman\\
\and
J.S. Love, trans.}


% Indicate main language of document is English
\setmainlanguage{english}
% Indicate language of source document is Swedish
\setotherlanguage{swedish}
% End of header

% Beginning of document content
\begin{document}
% Apply title and author information given in header
\maketitle

% Introductory material dvided by section
% The sections below are only suggestions. They can be removed or replaced as desired.
\section*{Introduction}
This is an edition and translation of a ninteenth-century travel narrative. It
is intended to serve as an example for how historical texts might be translated
and distributed more openly, especially as drafts prior to
publication; cf. \href{https://github.com/ThorkellMoon/OpenTranslation}{my OpenTranslation project on Github}.
\par
This is only an extract to demonstrate the OpenTranslation system, and a more extensive translation will be available in future. N.b. This text is only a draft and not a finished product! Work is ongoing.

\section*{Edition}
In the left column is the original text copied from the source edition, which is the Projekt Runeberg digital text\footnote{\href{runeberg.org/eugenie/1/}} of \textit{Fregatten Eugenies resa omkring jorden åren 1851-1853}, Stockholm: Bonnier.


\section*{Translator's statement}
The translation below is meant to be a small contribution to the field of Hawai'ian studies, and its primary function is to render this account of the Eugenie's visit to Hawai'i accessible to English-speaking audiences. Some license has been taken throughout in word choice and syntax to make it more readable in English.

\section*{Introduction}
Part of the Eugenie's world tour included a brief stop in the Hawai'ian Islands during the summer of 1852. [further details t.b.d]. Need to find out why the parcolumns package cannnot handle styling of long sentences and/or find out how they should be encoded.

% Here begins the section for source text and translated text
\section*{The Voyage of the Frigate Eugenie Around the World during the years 1851-1853}

\begin{parcolumns}{2}
\colchunk{IX. Honolulu}
\colchunk{IX. Honolulu}
\colplacechunks

\colchunk{1852. Juni 22 — Juli 2.}
\colchunk{1852. 22 June — 2 July}
\colplacechunks

\colchunk{[p184] Genast efter ankringen afsändes en officer i land för att uppgöra om
salut, och vid hans återkomst ombord saluterades Hawaiiska flaggen
med tjugueft skott, som af batteriet på toppen af den bakom
staden belägna Punchbowl-hill med samma antal besvarades. Kort
derefter kom en båt ombord, medförande en Hawaiisk officer, med
uppdrag att önska fregatten välkommen till Honolulu. Officern var en
reslig och starkt byggd man af godt utseende och var iklädd en
uniformsfrack med stora guldepåletter, hvita benkläder, värja och en
mycket galonerad hatt. Han talade något litet engelska, och var
ofantligt gladlynt och vänlig, tog alla menniskor i hand och frågade de på
post stående marin-soldaterna hur det stod till med deras helsa,
hvarvid dessa sågo helt förbluffade ut. Många båtar och kanoter infunno
sig snart, medförande en massa karlar, hvars ärende var att erbjuda
sig ombesörja tvätt för befälet.}
\colchunk{Immediately after anchoring an officer was sent on land to deliver a salute.
On his return onboard the Hawaiian flag was saluted with 21 shots, which were answered in the same measure by the battery lying behind the city atop Punchbowl Hill.
Shortly afterwards a boat came aboard bearing a Hawaiian officer with a charge to wish the frigate welcome to Honolulu.
The officer was a towering, strongly built man of good complexion and was dressed in a uniform coat with large gold epaulettes, white trousers, a blade and much galooned hat.
He spoke a bit of English and was immensely cheerful and friendly.
He took everyone by the hand and asked each of the marines standing at attention after their health, at which they were all quite amazed.
Many boats and canoes soon appeared, bearing a crowd of men whose purpose was to offer to ?take care of washing for the commanders.}
\colplacechunks

% Each set of colchunks represents a unit (e.g. a paragraph) of parallel text
% \colchunk{[insert source text]}
% \colchunk{[insert translated text]}
% \colplacechunks

% End of translated text
\end{parcolumns}


% End of document
\end{document}
